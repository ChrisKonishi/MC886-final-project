%% Adaptado de 
%% http://www.ctan.org/tex-archive/macros/latex/contrib/IEEEtran/
%% Traduzido para o congresso de IC da USP
%%*****************************************************************************
% Não modificar

\documentclass[twoside,conference,a4paper]{IEEEtran}

%******************************************************************************
% Não modificar
\usepackage{IEEEtsup} % Definições complementares e modificações.
\usepackage[utf8]{inputenc} % Disponibiliza acentos.
\usepackage[english,brazil]{babel}
%% Disponibiliza Inglês e Português do Brasil.
\usepackage{latexsym,amsfonts,amssymb} % Disponibiliza fontes adicionais.
\usepackage{theorem} 
\usepackage[cmex10]{amsmath} % Pacote matemático básico 
\usepackage{url} 
%\usepackage[portuges,brazil,english]{babel}
\usepackage{graphicx}
\usepackage{amsmath}
\usepackage{amssymb}
\usepackage{color}
\usepackage[pagebackref=true,breaklinks=true,letterpaper=true,colorlinks,bookmarks=false]{hyperref}
\usepackage[tight,footnotesize]{subfigure} 
\usepackage[noadjust]{cite} % Disponibiliza melhorias em citações.
%%*****************************************************************************

\begin{document}
	\selectlanguage{brazil}
	\renewcommand{\IEEEkeywordsname}{Palavras-chave}
	
	%%*****************************************************************************
	
	\urlstyle{tt}
	% Indicar o nome do autor e o curso/nível (grad-mestrado-doutorado-especial)
	\title{Título do Trabalho}
	\author{%
		\IEEEauthorblockN{Nome do autor\,\IEEEauthorrefmark{1}}
		\IEEEauthorblockA{\IEEEauthorrefmark{1}%
			Ciência da Computação - Graduação \\
			E-mail: \ldots@ic.unicamp.br}
	}
	
	%%*****************************************************************************
	
	\maketitle
	
	%%*****************************************************************************
	% Resumo do trabalho
\begin{abstract}
	O resumo deve conter uma breve descrição sobre várias partes do seu trabalho que serão tratadas no decorrer do texto. Primeiramente, pode-se descrever brevemente o problema no qual você está trabalhando: Por que você está desenvolvendo este trabalho? Qual a motivação para este desenvolvimento? Por que ele é importante? O resumo deve conter também um breve descritivo da metodologia que você usou no desenvolvimento: Que problema foi tratado? Como a solução foi construída/desenvolvida? Quais as tecnologias utilizadas? Finalmente, deve falar um pouco sobre os resultados que você conseguiu: o resultado final ficou bom? Quais os seus principais diferenciais? Qual a eficiência do desenvolvimento?
\end{abstract}

	
	% Indique três palavras-chave que descrevem o trabalho
	\begin{IEEEkeywords}
		Palavras-chave
	\end{IEEEkeywords}
	
	%%*****************************************************************************
	% Modifique as seções de acordo com o seu projeto
	
	\section{Introdução}

Na introdução você deve descrever os aspectos mais relevantes sobre a revisão bibliográfica que fez e do problema que você decidiu tratar. Quais foram os pontos estudados/pesquisados? Quais os outros trabalhos similares ao seu que você encontrou? 

Também na introdução espera-se que você descreva um pouco sobre a motivação de trabalhar com esse tema. A descrição do seu trabalho será feita em detalhes nas próximas seções do artigo.


No final da introdução, é comum inserir um parágrafo descrevendo o que será encontrado em cada seção no restante do seu texto. Exemplo: Este trabalho encontra-se organizado da seguinte forma: a seção 2 apresenta X. A seção 3 descreve Y. Os resultados são apresentados na seção 4, e as conclusões são apresentadas na seção 5.
	%Na hora de criar uma seção, crie um novo arquivo e coloque o input, para evitar deixcar tudo aqui
	
	\section{Seções}
	
	Utilize outras seções e subseções para abordar o problema. 
	
	\subsection{Uma subseção}
	
	Se precisar, você pode usar listas, tais como
	
	\begin{itemize}
		\item Item 1
		\item Item 2
	\end{itemize}
	ou
	\begin{enumerate}
		\item Item 1
		\item Item 2
	\end{enumerate}
	
	\section{Trabalho Proposto}
	
	Nesta seção descreva de forma abrangente, porém clara e organizada, o seu trabalho.
	
	\subsection{Tabelas}
	
	Uma tabela pode ser posicionada em qualquer lugar no texto, como no exemplo
	seguinte.
	%
	\begin{table}[ht]
		\renewcommand{\arraystretch}{1.3}
		\centering
		\caption{Exemplo de texto de uma tabela.}
		\label{tab:tab1}
		\begin{tabular}{lcccc}\hline
			& \multicolumn{2}{c}{Texto}
			& \multicolumn{2}{c}{Sem \#21} \\ \cline{2-5}
			X & Y & $z$ & $\mathcal{A}$ & valor-$z$ \\ \hline \hline
			1      &0,491  & 3,66   &0,367 &2,46  \\
			2    &0,732  & 4,21   &0,354 &1,50  \\
			3      &0,000  & -      &0,000 & -    \\
			4      &0,000  & -      &0,000 & -  \\
			5      &0,421  & 1,94   &0,668 &2,79  \\
			6      &0,421  & 1,94   &0,668 &2,79  \\
			7      &0,938  & 3,92   &1,295 &4,67 \\
			8       &0,000  & -      &0,000 & - \\
			9       &0,356  & 1,40   &0,491 &1,87 \\ \hline
		\end{tabular}
	\end{table}
	
	Para citar esta tabela, em qualquer ponto no texto, como Tabela~\ref{tab:tab1}.
	
	
	\section{Materiais e Métodos}
	
	Todo trabalho deve ser submetido a algum tipo de teste para que possa ser avaliado. Na verdade, buscamos aqui uma validação com um caráter mais científico de seu trabalho (validação de hipótese). Busca-se identificar quais os seus pontos fortes e fracos. Nesta seção você deve descrever claramente quais foram e como foram conduzidos os testes, quais os materiais e as metodologias empregadas.   
	
	Uma figura pode ser posicionada em qualquer lugar no texto, como no exemplo seguinte da Figura~\ref{fig:fig1}.
	
	\begin{figure}[ht]
		\centering
		\includegraphics[width=1\hsize]{figuras/pipeline.png}
		\caption{Um exemplo de figura.}
		\label{fig:fig1}
	\end{figure}
	
	Use o comando ``cite'' para citar itens na sua lista de
	referências através dos seus rótulos. Exemplo: \cite{Rowling:1997}\cite{Reynolds:2009a}\cite{Michalowski:2006}.
	
	
	\section{Resultados e Discussão}
	
	Nesta seção você deve apresentar claramente os resultados obtidos para os testes efetuados. Procure organizar os dados utilizando uma linguagem científica. Algumas opções são o uso de tabelas e gráficos, para que a compreensão seja fácil e rápida. 
	
	\section{Conclusões}
	
	Nesta seção, faça uma análise geral de seu trabalho, levando em conta todo o processo de desenvolvimento e os resultados. Quais os seus pontos fortes? Quais os seus pontos fracos? Quais aspectos de sua metodologia de trabalho foram positivas? Quais foram negativas? O que você recomendaria (ou não recomendaria) a outras pessoas que estejam realizando trabalhos similares aos seus? 
	
	
	%******************************************************************************
	% Referências - Definidas no arquivo Relatorio.bib
	+-------------+
	
	\bibliographystyle{IEEEtran}
	
	\bibliography{Relatorio}
	
	
	%******************************************************************************
	
\end{document}
